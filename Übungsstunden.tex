\documentclass[a4paper,10pt]{article}
\usepackage{graphicx} % Required for inserting images
\usepackage[nswissgerman]{babel}
%4 stackanchor  
\usepackage{stackengine}
% define nice looking boxes
\usepackage[many]{tcolorbox}

% a base set, that is then customised
\tcbset {
  base/.style={
    boxrule=0mm,
    leftrule=1mm,
    left=1.75mm,
    arc=0mm, 
    fonttitle=\bfseries, 
    colbacktitle=black!10!white, 
    coltitle=black, 
    toptitle=0.75mm, 
    bottomtitle=0.25mm,
    title={#1}
  }
}
\definecolor{brandblue}{rgb}{0.34, 0.7, 1}
\newtcolorbox{mainbox}[1]{
  colframe=brandblue, 
  base={#1}
}

\definecolor{orange}{rgb}{1, 0.55, 0.3}
\newtcolorbox{tbox}[1]{
  colframe=orange, 
  base={#1}
}

\definecolor{green}{rgb}{0.294, 0.729, 0.254}
\newtcolorbox{bembox}[1]{
  colframe=green, 
  base={#1}
}

\definecolor{red}{rgb}{0.99, 0.04, 0.99}
\newtcolorbox{tipbox}[1]{
  colframe=red, 
  base={#1}
}

\newtcolorbox{defbox}[1]{
  colframe=black!20!white,
  base={#1}
}
% Mathematical typesetting & symbols
\usepackage{amsthm, mathtools, amssymb} 
\usepackage{marvosym, wasysym}


\allowdisplaybreaks

% Tables
\usepackage{tabularx, multirow}
\usepackage{booktabs}
\renewcommand*{\arraystretch}{2}

% Make enumerations more compact
\usepackage{enumitem}
\setitemize{itemsep=0.5pt}
\setenumerate{itemsep=0.75pt}

% To include sketches & PDFs
\usepackage{graphicx}

% For hyperlinks
\usepackage{hyperref}
\hypersetup{
  colorlinks=true
}
% Math helper stuff
\def\limn{\lim_{n\to \infty}}
\def\limxo{\lim_{x\to 0}}
\def\limxi{\lim_{x\to\infty}}
\def\limxn{\lim_{x\to-\infty}}
\def\sumk{\sum_{k=1}^\infty}
\def\sumn{\sum_{n=0}^\infty}
\def\R{\mathbb{R}}
\def\C{\mathbb{C}}
\def\dx{\text{ d}x}
\usepackage[utf8]{inputenc}

\title{Übungsstunden}
\author{Konstantin Lucny}
\date{HS 2023}
\begin{document}
\maketitle

\section{Übungsstunde}
\section{Übungsstunde}
\subsection{Lineare ODEs Ordnung 1 Beispiele}
\begin{tipbox}
    {Anleitung Lösung ODEs 1. Ordnung}
     $y'+a(x)y=b(x)$
    \begin{enumerate}
        \item $f_1(x)$ Lösung des Homogenen Problems $y'+a(x)y=0$\\[3pt]
        $f_1$ als Basis der homogenen Lösung $f_1=exp(-A(x)),\ A(x)=\int a(x)$\\[3pt]
        (falls $b(x)=0, f=c\cdot f_1(x), c\in \R$)
        \item $f_0(x)$ partikulär Lösung von $y'+a(x)y=b(x)$\\[3pt]
        $f_0(x)=z(x)\cdot \exp{(-A(x))}$, $z(x)$ primitive von $b(x)\cdot \exp{(A(x))}, \ z(x)=\int b(x)\exp{(\int a(x))}dx$
        \item $y=f_0+c_1f, \ c_1 \in \R$
        \item Kontrollieren
        \item Anfangswertproblem: $y(0)=0; \ y(x_0)=y_0$
    \end{enumerate}
\end{tipbox}
\begin{itemize}
    \item Beispiel: $y'+\frac{1}{x}y=x^2+4$
    \begin{enumerate}
        \item $A(x)=\int \frac{1}{x}dx = \ln(x)$\\[1pt]
        $y'+\frac{1}{x}=0$\\[1pt]
        $f_1=\exp(-A(x))=e^{-\ln(x)}=\frac{1}{x}$
        \item $A(x)=\frac{1}{x}$\\[1pt]
        $z(x)=\int(x^2+4)x \dx=\int x^3 +4x \dx =\frac{1}{4}x^4+2x^2$\\[1pt]
        $z_0=(\frac{1}{4}x^4+2x^2)\cdot \frac{1}{x}=\frac{1}{4}x^3+2x$
        \item $y=\frac{1}{4}x^3+2x+c_1\cdot\frac{1}{x}, c_1\in\R$
        \item $(\frac{1}{4}x^3+2x+\frac{c_1}{x})'+\frac{1}{x}(\frac{1}{4}x^3+2x+\frac{c_1}{x})=\frac{3}{4}x^2+2-\frac{c_1}{x^2}+\frac{1}{4}x^2+2+\frac{c_1}{x^2}=x^2+4$
    \end{enumerate}
    \pagebreak
    \item Beispiel: $y'-y=x$
    \begin{enumerate}
        \item $A(x)=\int -1 \dx= -x$\\[1pt]
        $f_1=\exp(-(-x))=e^x$
        \item $z(x)=\int x\cdot e^{-x} \dx=-xe^{-x}-\int e^-x\dx=-xe^{-x}-e^{-x}=
        \\-(x+1)e^{-x}$\\[1pt]
        $f_0=-(x+1)e^{-x}e^x=-(x+1)$
        \item $y=-(x+1)+c_1e^x, c_1\in\R$\\[1pt]
        \textbf{Überprüfung}: $(-(x+1)+c_1e^x)'+(x+1)-c_1e^x=-1+c_1e^x+x+1-c_1e^x=x$
        \item Wenn Anfangswert gegeben ist: $x_0=0, y_0=0, y(0)=0$\\[1pt]
        $\implies 0-(0+1)+c_1e^0=0\\
        \iff -1+c_1=0\\
        \iff c_1 = 1$\\[1pt]
        $y=-(x+1)+e^x$
    \end{enumerate}
\end{itemize}
\subsection{Lineare ODE's mit Konstanten Faktoren}
$y^{(k)}+a_{k-1}y^{(k-1)}+...+a_1y'+a_0y=b(x)\implies a_0,...,a_{k-1}$ konstante, $\in\R$\\[1pt]
\begin{tipbox}
    {Theorie: Lineare ODE's Ordnung $k\ge1$}
    \begin{enumerate}
        \item Lösung des Homogenen Problems $(b(x)=0)$\\[1pt]
        Suche $f_1,...,f_k$ Lösungen die einen \underline{k-dimensionalen} Raum strecken.
        \item Suchen eine partikulär Lösung für $b(x)\neq 0$ also
        $f_0$
        \item $y=f_0+\sum_{i=1}^kc_if_i$ $(c_1,...,c_k\in\R)$
        \item Bei Anfangswertproblem: $y(x_0)=y_0,...,y^{(k-1)}(x_0)=y_{k-1}$
    \end{enumerate}
\end{tipbox}
\begin{tbox}
    {Theo 2.2.3: Lineare ODE's mit $a_0,...,a_{k-1},b$ stetig}
    Vorraussetzung: $a_0,...,a_{k-1},b$ stetig
    \begin{enumerate}
        \item Lösungen der Homogenen Gleichung bildet einen \underline{k-dimensionalen} Vektorraum
        \item Lösung des inhomogenen Problems ist der affine Raum: $f_0+\sum c_i f_i$
        \item Für $k$ Anfangswerte ist die Lösung eindeutig
    \end{enumerate}
\end{tbox}
\begin{bembox}
    {Grad einer Gleichung}
    ODE's Ordnung $k\ge 2$ kann man in ein System von \underline{k Gleichungen} umwandeln mit \underline{Grad 1}\\[2pt]
    \textbf{Bsp.:} $y''+y'=3\ (z_0=y, z_1=y')\implies
    \begin{cases}
        z_1'+z_1=3\\
        z_0'=z_1
    \end{cases}$\\[2pt]
    $z_0''+z_0'=3$
\end{bembox}
\textbf{Woher kommt $f_1$ eigentlich?}\\
$f_1=\exp(-A(x))$\\[2pt]
$y'+a(x)y=0$\\[2pt]
$\implies y'+ay=0 \iff \frac{y'}{y}=-a\iff \int \frac{y'}{y} \dx =\int -a\dx \iff \ln(y)=-A(x)\iff y=\exp(-A(x))$ 
\end{document}