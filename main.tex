\documentclass[a4paper,10pt]{article}
\usepackage{geometry}
\geometry{margin=1.2in}
\usepackage{graphicx} % Required for inserting images
\usepackage[nswissgerman]{babel}
%4 stackanchor  
\usepackage{stackengine}
% define nice looking boxes
\usepackage[many]{tcolorbox}
% a base set, that is then customised
\tcbset {
  base/.style={
    boxrule=0mm,
    leftrule=1mm,
    left=1.75mm,
    arc=0mm, 
    fonttitle=\bfseries, 
    colbacktitle=black!10!white, 
    coltitle=black, 
    toptitle=0.75mm, 
    bottomtitle=0.25mm,
    title={#1}
  }
}
\definecolor{brandblue}{rgb}{0.34, 0.7, 1}
\newtcolorbox{mainbox}[1]{
  colframe=brandblue, 
  base={#1}
}

\definecolor{orange}{rgb}{1, 0.55, 0.3}
\newtcolorbox{tbox}[1]{
  colframe=orange, 
  base={#1}
}

\definecolor{green}{rgb}{0.294, 0.729, 0.254}
\newtcolorbox{bembox}[1]{
  colframe=green, 
  base={#1}
}

\definecolor{red}{rgb}{0.99, 0.04, 0.99}
\newtcolorbox{tipbox}[1]{
  colframe=red, 
  base={#1}
}

\newtcolorbox{defbox}[1]{
  colframe=black!20!white,
  base={#1}
}
% Mathematical typesetting & symbols
\usepackage{amsmath, mathtools, amssymb}

\usepackage{marvosym, wasysym}


\allowdisplaybreaks

% Tables
\usepackage{tabularx, multirow}
\usepackage{booktabs}
\renewcommand*{\arraystretch}{2}

% Make enumerations more compact
\usepackage{enumitem}
\setitemize{itemsep=0.5pt}
\setenumerate{itemsep=0.75pt}

% To include sketches & PDFs
\usepackage{graphicx}

% For hyperlinks
\usepackage{hyperref}
\hypersetup{
  colorlinks=true
}
% Math helper stuff
\def\limn{\lim_{n\to \infty}}
\def\limxo{\lim_{x\to 0}}
\def\limxi{\lim_{x\to\infty}}
\def\limxn{\lim_{x\to-\infty}}
\def\sumk{\sum_{k=1}^\infty}
\def\sumn{\sum_{n=0}^\infty}
\def\R{\mathbb{R}}
\def\dx{\text{ d}x}
\usepackage[utf8]{inputenc}

\title{Vorlesungen}
\author{Konstantin Lucny}
\date{HS 2023}
\begin{document}
\maketitle
\section{Einführung}
\section{Gewöhnliche Differentialrechnung (ODE)}
\subsection{Einleitung (\today)}
$e^{y''(x)\cdot y(x)}+\sin(\sqrt{y})-3=0$\\
\begin{defbox}
{Definition: Differentialgleichung}
    \textbf{Definition}: Eine Gleichung der Form $G(y,y',...,y^{(k)},x)=0$ heisst gewöhnliche Differentialgleichung.\\
Eine \underline{Lösung} ist eine k-mal differenzierbare Funktion $f:I\rightarrow \mathbb{R}$ auf einem offenen Intervall $I$ mit $G(f(x),f'(x),...f^{(k)}(x), x)= 0$ für alle  $x\in I$.\\
\end{defbox}
Beispiele in Notizen \today\\
\textbf{Achtung}: 
\begin{itemize}
    \item Es darf \underline{nur} x bei y(x) sein und nichts anderes z.B. $y(\tan(x))$
    \item Es \underline{muss} die Ableitung in der Formel sein. Sonst kein OED
\end{itemize}Es darf \underline{nur} x bei y(x) sein und nichts anderes z.B. $y(\tan(x))$

\begin{defbox}
    {Definition: Ordnung}
    $k\ge 1$ heisst \underline{Ordnung} der ODE (ordinary differential equation)
\end{defbox}
Ist zu der Definition noch zusätzlich $y(x_0)=y_0, y'(x_0)=y_1,...,y^{(k-1)}(x_0)=y_{k-1}$ mit $x_0, y_1,...,y_{k-1}\in\mathbb{R}$ erfüllt, spricht man von einem \underline{Anfangspunktproblem}.\\
\begin{bembox}
    {Info: ODEs Verwendung}
    ODEs eignen sich zur \textbf{Modellierung} von z.B. 2. Newtonsche Gesetz: $y(x)=$ Position eines Körpers von Masse $m$ zur zeit $x$. Dann erfüllt $y$ die ODE $my''=$ Summe des auf den Körper wirkenden Kräfte. 
\end{bembox}
Sind beim obrigen Beispiel $y(x_0), y'(x_0)$ vorgegeben, dann liegt ein Anfangsproblem vor.
\begin{tipbox}
    {Lösungsmethode: Seperation der Variablen (Kap. 2.6)}
    ODE der Form $y'=\frac{1}{a(y)}\cdot b(x)$ mit $a,b$ stetig, \\
    $a(y)\neq0$
    \\ $\iff a(y)\cdot y'=b(x)$
    \\ $\iff \int a(y)\cdot y'(x)dx=\int b(x) dx +c$
    \\ $A(y)=B(x)+c$ wobei $A,B$ Stammfunktionen von $a,b \implies y = A^{-1}(B(x)+c)$ falls $A^{-1}$ eine Umkehrfunktion von $A$ ist.
\end{tipbox}
\textbf{Bsp}. zu obrigen Lösung\\
$y'=\begin{cases}
    0 & für\ y\le0\\
    -\sqrt{y} & sonst
    \end{cases}$\\
$a(y)=\frac{-1}{\sqrt{y}}$ falls $y>0$, $b(x)=1$
\\ $A(y)=-2\sqrt{y}, B(x)=x \implies -2\sqrt{y}=x+c\iff \sqrt{y}=-\frac{1}{2}(x+c)$
\\ $y=\frac{1}{4}(x+c)^2\implies$ Lösung: $f(x)=\frac{1}{4}(x+c)^2$ auf $(-\infty,-0)$
\begin{tbox}
    {Existenz \& Eindeutigkeitssatz (2.1.6)}
    \textbf{Kurze Def.}: Ein Anfangswertproblem $y'=F(y,x)$ $y(x_0)=y_0$ mit $F$ stetig differenzierbar, gibt es \underline{genau} eine \underline{maximale} Lösung.\\[10pt]
    \textbf{Detailierte Version:} Ist $F:\mathbb{R}^2\rightarrow \mathbb R $ in einer Umgebung von $(y_0,x_0)$ stetig ist, und nach $y$ differenzierbar ist, dann gibt es eine Lösung $f:I\rightarrow\mathbb R$ auf einem offenen $I$ mit $x_0\in I$, sodass für jede andere Lösung $g:J\rightarrow\mathbb R$ gilt: $J\subseteq I, f|_J=g$
\end{tbox}
\textbf{Beispiel} (zu 2.1.6): Siehe Notizen \today\\[5pt]
\textbf{Achtung}: Diese Definition gilt nur für die 1. Ordnung.
\end{document}